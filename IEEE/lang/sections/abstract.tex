\begin{abstract}
%The so-called model-driven engineering approach relies on two paradigms, abstraction and automation, recognized as very efficient for dealing with complexity of today system. 
%Abstraction is the ability to provide simplified and focused view of a system and requires adequate modeling language. 
%For this concern, it is clear that the Unified Modeling Language (UML) is nowadays the most used, educated, documented and tooled modeling language. 
%Even, if a graphical language such as the UML is not the silver bullet for all software related concerns, it provides hence better support than text-based solutions for some concerns such as architecture and logical behavior of application development. 
UML state machine and their visual representations are much more suitable to describe logical behaviors of system entities than any equivalent text based description such as IF-THEN-ELSE or SWITH-CASE constructions. Although many industrial tools and research prototypes can generate executable code from such graphical language, generated code could be manually modified by programmers. 
After code modifications, round-trip engineering is needed to make the model and code consistent, which is a critical aspect to meet quality and performance constraint required from project manager today. Unfortunately, current UML tools only support structural concepts for round-trip engineering such as those available from class diagrams. In this paper, we address the round-trip engineering of UML state-machine and its related generated code. We propose a round-trip engineering approach consisting of a forward process which generates code by using transformation patterns, and a backward process which is based on code pattern detection to update the original state machine model from the modified generated code. We implemented a prototype and conducted several experiments on different aspects of the round-trip engineering to verify the proposed approach.


%Unified Modeling Language (UML) State machine is widely used 

%Model-driven engineering (MDE) is a development methodology aiming to increase software productivity and quality by automatically generating code from higher level abstraction models. 
%A recent survey has revealed that industries are gaining the adoption of code generation into software development life-cycle. 
%Although many tools and research prototypes can generate executable code from models (e.g. Unified Modeling Language), developers, after code generation, tend to abandon models and code generators in software evolution. The reason behind is that generated code could be manually modified by developers and code modifications are not easily propagated back to models. Round-trip engineering (RTRIP) supported by many tools is needed to make the model and code consistent but most of the tools are only applicable to static diagrams such as classes. In this paper, we tackle a classical problem : from UML State Machine diagrams to code and back. We propose a RTRIP approach consisting of a forward process, which generates code, and a backward process, which updates the original state machine from the modified generated code. From the proposed approach, we implemented a prototype and conducted several experiments on different aspects of the round-trip engineering to verify the proposed approach.
\end{abstract}