\section{Related Work}
\label{sec:related_works}
Our work is motivated by the desire to reduce the gap between and synchronize artifacts at different levels of abstraction, model and code in particular, in developing reactive systems. Specifically, the usage of USMs for describing the logical behavior of complex, reactive systems is indispensable. In the following sections, we compare our approach to related topics recorded in the literature.
\subsection{Implementation and code generation techniques for USMs}
Implementation and code generation techniques for USMs are closely related to the forward engineering of our RTE.
%Main approaches including switch/if, state table and state pattern are investigated.

%Switch/if is the most intuitive technique implementing a "flat" state machine. Two types of switch/if are supported. The first one uses a scalar variable representing the current active state \cite{Booch1998}. A method for each event processes the variable as a discriminator in switch/if statement. The second one uses a double nested switch/if and has two variables to represent the current active state and the event to be processed \cite{Douglass1999}. The latter are used as the discriminators of an outer switch statement to select between states and an inner one/if statement to decide how the event should be processed. The behavior code of the two types is put in one file or class. This practice makes code cumbersome, complex, difficult to read and less explicit when the number of states grows or the state machine is hierarchical. Furthermore, the first approach lets the code scatter in different places. Therefore, maintaining or modifying such code of complex systems is very difficult.

Switch/if is the most intuitive technique implementing a "flat" state machine. The latter can be implemented by either
using a scalar variable \cite{Booch1998} and a method for each event or using two variables as the current active state and the incoming event used as the discriminators of an outer switch statement to select between states and an inner one/if statement, respectively. The double dimensional state table approach \cite{Douglass1999} uses one dimension represents states and the other one all possible events. 
%Each cell of the table is associated with a function pointer meaning that the state associated with a dimension index of the cell is triggered by the event associated with the other dimension. 
The behavior code of these techniques is put in one file or class. This practice makes code cumbersome, complex, difficult to read and less explicit when the number of states grows or the state machine is hierarchical. 
%Therefore, maintaining or modifying such code of complex systems is very difficult. 
Furthermore, these approaches requires every transition must be triggered by at least an event. This is obviously only applied to a small sub-set of USMs.  
  
State pattern \cite{Shalyto2006,Douglass1999} is an object-oriented way to implement flat state machines. Each state is represented as a class and each event as a method. %The event is processed by a delegation from the context class to sub-states. 
Separation of states in classes makes the code more readable and maintainable. %Unfortunately, this technique only supports flat state machines. 
This pattern is extended in \cite{niaz_mapping_2004} to support hierarchical-concurrent USMs. However, the maintenance of the code generated by this approach is not trivial since it requires many small changes in different places. %This is impractical when dealing with large state machines. %Furthermore, similar to the state table, this approach also poses the requirement of having at least one event for transition.

Many tools, such as \cite{ibm_rhapsody, sparxsystems_enterprise_2014}, apply these approaches to generate code from USMs. Readers of this paper are recommended referring to \cite{Domnguez2012} for a systematic survey on different tools and approaches generating code from USMs.

Double-dispatch (DD) pattern in \cite{spinke_object-oriented_2013} in which %as a new technique to implement state machines. 
represent states and events as classes. Our generation approach relies on and extends this approach. The latter profits the polymorphism of object-oriented languages. %provides some 1-1 mappings from state machines to object-oriented code and the implementation technique 
%is not dependent on a specific programming language. 
However, DD does not deal with triggerless transitions and different event types supported by UML such as \ti{CallEvent}, \ti{TimeEvent} and \ti{SignalEvent}. Furthermore, DD is not a code generation approach but an approach to manually implementing state machines.

\subsection{Round-trip engineering}
Our RTE is related to synchronization of model-code and models themselves that a large number of approaches support. This paper only shows the most related approaches. 

\noindent
\ti{Model-code synchronization}

Commercial and open-source tools such as \cite{sparxsystems_enterprise_2014, ibm_rhapsody} only support RTE of architectural model elements and code.
Systematic reviews of some of these tools are available in \cite{Cutting2015}.

% RTE restriction
Some RTE techniques restrict the development artifact to avoid synchronization problems.
Partial RTE and protected regions \cite{Frankel:2002:MDA:579151} aim to preserve code changes which cannot be propagated to models.
This approach separates the code regions that are generated from models
from regions which are allowed to be edited by developers. 
This form of RTE is unidirectional only and does not support iterative development \cite{Jorges2013}
Our approach does not separate different regions but supports a semantic code analysis in the reverse direction.

Fujaba \cite{KNNZ99_2_ag} offers an RTE environment. An interesting part of Fujaba is that it abstracts from Java source code to UML class diagrams and a so-called story-diagrams. Java code can also be generated from these diagrams. RTE of these diagrams and code works but limited to the naming conventions and implementation concepts of Fujaba which are not UML-compliant. 

\noindent
\ti{Model synchronization}

RTE of models is tackled by many approaches categorized by its model transformation from total, injective, bi-directional to partial non-injective transformations \cite{Hettel2008}. %The authors in \cite{Paesschen2005} propose an injective mapping of elements in the source model to
%the target model. 
Techniques and technologies, such as Triple Graph Grammar (TGG) \cite{giese_incremental_2006}
and QVT-Relation \cite{Omg2008},
allow synchronization between source and target elements that have non-injective mappings.
These techniques require a mapping model to connect the source and target models which need to be persisted in a model store \cite{Bergmann2011}.
Mappings between USMs and code in our approach are non-bijective. We only use simple tables for storing tracing information.

%Foster et al \cite{foster_combinators_2007} uses the concepts of lenses to define RTE. A lens consists of a \ti{get} function, which produces a target artifact (model/code) from a source artifact (model/code), and a \ti{putback} function, which takes as input the modified target and old source artifact to update the source artifact. The paper also defines two RTE laws that a RTE must satisfy. These laws are the base for the evaluation of our approach that will be presented in Section \ref{sec:experiments}.  
 

%\ti{Partial RTE} and \ti{protected regions} are introduced in \cite{Frankel:2002:MDA:579151} and \cite{Kelly2007} to preserve code modifications which cannot be reflected to models. This approach separates the code part which is generated from models and the part which is allowed to be changed by programmers. This form of RTE is unidirectional only and does not support iterative development \cite{Jorges2013}. 

%RTE of UML models and object-oriented code are supported in many tools including research prototypes and commercial such as Enterprise Architect \cite{sparxsystems_enterprise_2014}, Visual Paradigm \cite{visual}, AndroMDA \cite{_andromda_}. Although these tools work well with UML class diagrams and code, behavioral diagrams are not well supported.  

%The authors in \cite{angyal_synchronizing_2008} propose a syntactic synchronization technique for domain-specific modeling languages (DSML) and code. The approach uses an Abstract Syntax Tree (AST) metal-model to model source. Changes in code detected by using an algorithm proposed  in \cite{Chawathe:1996:CDH:235968.233366}. %to compare the AST instance of the current code with the last synchronized one are merged to the instance of DSML. 
%However, an AST is very low level and it is not clear to have mappings from DSML instances to AST instances. Furthermore, although there is an example for illustrating the technique, a systematic evaluation of the approach is not presented to show its scalability.


%Other techniques \cite{foster_combinators_2007}
%are based on bi-directional transformations, which comprise a forward transformation a backward transformation.
%Bi-directional transformations provides a novel mechanism for synchronization.
%Some works \cite{Matsuda2015} derive a backward transformation based on forward
%transformation.
%However, such works do not offer any means to synchronize models that are concurrently edited.

%A few approaches derive model synchronization from model transformation while allowing concurrent editions
%of both source and target models.
%In \cite{xiong_towards_2007} the authors propose to automatically derive
%model synchronization of a source and a target model related by an ATL \cite{eclipse_foundation_eclipse_2016}
%model transformation.
%The synchronization is based on differentiating source and target model states.
%But, reflectable addition of an element in the target model is not well handled according to \cite{xiong_towards_2007}.
Differently from other approaches, ours is specific to RTE of USMs and code. The goal is to provide a full model-code synchronization supporting for rapidly, iteratively, and efficiently developing reactive systems. 