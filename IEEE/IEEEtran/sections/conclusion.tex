\section{Conclusion}
\label{sec:conclusion}
This paper presented a novel approach to RTE from USMs to code and back. The forward process of the approach is based on different patterns transforming USM elements into an intermediate model containing UML classes. Object-oriented code is then generated from the intermediate model by existing code generators. In the backward direction, code is analyzed and transformed into an intermediate whose format is close to the semantics of USMs. USMs are then reconstructed or updated from the intermediate format. 
 
The paper also showed the results of several experiments on different aspects of the proposed RTE with the tooling prototype. Specifically, the experiments on the correctness, semantic conformance of code, and the cost of system development/maintenance using the proposed RTE are conducted. Although, the reverse direction only works if manual code is written following pre-defined patterns, the semantics of USMs is explicitly present in generated code and easy to follow/modify.

While the semantic conformance of code generated is critical, the paper only showed a lightweight experiment on this aspect. A systematic evaluation is therefore in future work. We will also compare our code generation approach with commercial tools such as Rhapsody and Enterprise Architect. Furthermore, as evaluated in [7], the approach inheriting from the double-dispatch trades a reversible mapping for a slightly larger overhead. For the moment, the approach does not support RTE of concurrent state machines and several pseudo-states. Hence, future work should resolve these issues.
